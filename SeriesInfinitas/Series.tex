% ****************************************************************************************
% ************************     	SERIES INFINITAS	  ************************************
% ****************************************************************************************


% =======================================================
% =======	ALL COMMANDS AND RULES FOR DOC 	 ============
% =======================================================
\documentclass[12pt]{report}							    %Type of docuemtn and size of font
\usepackage[margin=1.2in]{geometry}							%Margins

\usepackage[spanish]{babel}									%Please use spanish
\usepackage[utf8]{inputenc}									%Please use spanish	
\usepackage[T1]{fontenc}									%Please use spanish

\usepackage{amsthm, amssymb, amsfonts}				        %Make math beautiful
\usepackage[fleqn]{amsmath}                                 %Please make equations left
\decimalpoint												%Make math beautiful
\setlength{\parindent}{0pt}									%Eliminate ugly indentation

\usepackage{graphicx}										%Allow to create graphics
\usepackage{wrapfig}                                    	%Allow to create images
\graphicspath{ {Graphics/} }                                %Where are the images :D
\usepackage{listings}										%We will be using code here
\usepackage[inline]{enumitem}								%We will need to enumarate

\usepackage{fancyhdr}										%Lets make awesome headers/footers
\renewcommand{\footrulewidth}{0.5pt}						%We will need this!
\setlength{\headheight}{16pt} 								%We will need this!
\setlength{\parskip}{0.5em}									%We will need this!
\pagestyle{fancy}											%Lets make awesome headers/footers
\lhead{\footnotesize{\leftmark}}							%Headers!
\rhead{\footnotesize{\rightmark}}							%Headers!
\lfoot{Compilando Conocimiento}								%Footers!
\rfoot{Oscar Rosas}						                    %Footers!

\author{Oscar Andrés Rosas}						            %Who I am




% =====================================================
% ============     	  COVER PAGE	   ================
% =====================================================
\begin{document}
\begin{titlepage}

	\center
	% ============ UNIVERSITY NAME AND DATA =========
	\textbf{\textsc{\Large Proyecto Compilando Conocimiento}}\\[1.0cm] 
	\textsc{\Large Calculo}\\[1.0cm] 

	% ============ NAME OF THE DOCUMENT  ============
	\rule{\linewidth}{0.5mm} \\[1.0cm]
		{ \huge \bfseries Series Infinitas}\\[1.0cm] 
	\rule{\linewidth}{0.5mm} \\[2.0cm]
	
	% ====== SEMI TITLE ==========
	{\LARGE Análisis de Series Infinitas: Convergencia y Divergencia}\\[7cm] 
	
	% ============  MY INFORMATION  =================
	\begin{center} \large
	\textbf{\textsc{Autor:}}\\
	Rosas Hernandez Oscar Andres
	\end{center}

	\vfill

\end{titlepage}

% =====================================================
% ========                INDICE              =========
% =====================================================
\tableofcontents{}
\clearpage

% ======================================================================================
% ==================================     SERIES    =====================================
% ======================================================================================
\chapter{Tipos de Series}
    \clearpage

    % =====================================================
    % ========         SERIES GEOMETRICAS           =======
    % =====================================================
    \section{Series Geométricas}
        Son series del estilo $a + ar + ar^2 + ar^3 + \cdots$, podemos
        generalizarlas como:

        \begin{equation}
            \sum_{n=1}^{\infty} ar^{n-1} = \sum_{n=0}^{\infty} ar^n 
        \end{equation}

        \textbf{Recuerda:}
        Podemos saber facilmente si converge o no, solo basta con que $|r| < 1$ 
        para estar seguros de que converge, donde podemos encontrar a que convege
        también muy fácil como:

        \begin{equation}
            \sum_{n=1}^{\infty} ar^{n-1} = \frac{a}{1-r}
        \end{equation}

        De no ser así, es decir, si $|r| \geq 1$ podemos estar seguros de que diverge.


        % ==========================
        % ====    EJEMPLO  =========
        % ==========================
        \subsection{Ejemplo}
        Un ejercicio muy sencillo es ver a que converge la siguiente sucesión:
        \begin{equation*}
            5 - \frac{10}{3} + \frac{20}{9} - \frac{40}{27} + \cdots
        \end{equation*}

        Podemos encontrar la respuesta facilmente porque vemos que
        $r=-\frac{2}{3}$ y como $|r|<1$ la Suma es:
        \begin{equation*}
            \frac{5}{1-\frac{2}{3}} = \frac{5}{\frac{5}{3}} = 3
        \end{equation*}



    % =====================================================
    % ========              SERIES - P            =========
    % =====================================================
    \clearpage
    \section{Series P: La Madre de todas las Armónicas}
        Para empezar hay que recordar que hay una serie muy famosa que se conoce como
        la Serie Armónica:
        \begin{equation}
            \sum_{n=1}^{\infty} \frac{1}{n} = 1 + \frac{1}{2}  + \frac{1}{3} + \cdots
        \end{equation}

        Podemos entonces hablar de las Series P, que es una generalización de las series
        armonicas, de la forma:

        \begin{equation}
            \sum_{n=1}^{\infty} \frac{1}{n^P}
        \end{equation}

        \textbf{Recuerda:}
        \begin{itemize}
            \item Cuando $p\leq1$ es la serie armónica (La cual diverge).
            \item Y tambien podemos saber (por el criterio de la integral) que para
            cualquiera $p > 1$ la serie converge.
        \end{itemize}


    % =====================================================
    % ========          SERIES TELESCOPICAS         =======
    % =====================================================
    \clearpage
    \section{Series Telescópicas}

        Las series telescópicas son muy lindas, para empezar lo que tenemos que
        hacer es ver que la Serie (Suma de todos los elementos de la Sucesión)
        tiene esta forma:

        \begin{equation}
            \sum_{n=1}^{\infty} b_n = (b_1-b_2) + (b_2-b_3) +  \cdots + (b_n-b_{n+1})
        \end{equation}

        O de manera mas concreta como:
        \begin{equation}
            \sum_{n=1}^{\infty} (b_{n} - b_{n+1})
        \end{equation}

        Y si te das cuenta todo eso se cancela, menos dos elementos, por lo podemos
        escribir así:
        \begin{equation}
            S_n = b_1 - b_{n+1}
        \end{equation}

        Y por lo tanto podemos ver que la serie (el límite de n en el
        infinito de las sumas parciales) es:

        \begin{equation}
            \sum_{n=1}^{\infty} b_n =  b_1 - \lim_{n \to \infty} b_{n+1}
        \end{equation}




    % =====================================================
    % ========          SERIES ALTERNANTES          =======
    % =====================================================
    \clearpage
    \section{Series Alternantes}

        Son un tipo de serie muy especial en la cual el signo cambia con cada
        termino. Las llamamos como serie alternante porque sus terminos alternan
        entre positivos y negativos.

        Podemos ver aquí que hay dos tipos de Series Alternantes:
        \begin{itemize}
            \item Si empezamos con números positivos es del tipo
                $\sum_{n=1}^{\infty} (-1)^{n-1} b_n$
            
            \item Si empezamos con números negativos es del tipo
            $\sum_{n=1}^{\infty} (-1)^n b_n$
        \end{itemize}

        Donde es bastante obvio que $b_n = |a_n|$


    \subsection{Estimación para Series Alternas}
        Una suma parcial de de cualquier serie convergente se puede usar como una
        aproximación a una suma total, pero no es muy utilizado, a menos que estime
        la exactitud de la aproximación.

        Esto es de verdad muy útil con las Series Alternantes, supongamos una Serie
        convergente, donde podemos escribir la Suma Parcial como $S = \sum (-1)^{n-1}b_n$
        que cumple con que:
        \begin{itemize}
            \item $ 0 \leq b_{n+1} \leq b_n$
            \item $ \lim_{n \to \infty} b_n = 0$
        \end{itemize}

        Entonces podemos decir que nuestra estimación será:
        \begin{equation}
            |R_n| = |S - S_n| \leq b_{n+1}
        \end{equation}







% ======================================================================================
% ===============================   CRITERIOS      =====================================
% ======================================================================================
\clearpage
\chapter{Criterios en Series}

    % =====================================================
    % ========       PRUEBA DE DIVERGENCIA        =========
    % =====================================================
    \clearpage
    \section{Prueba de la Divergencia}
        Esta es muy clásica y es muy fácil primero ahcer esta antes
        de hacer nada más:

        \begin{itemize}
            \item \textbf{Original} Si la Serie $\sum_{n=1}^{\infty} a_n$ es
            convergente entonces $\lim_{n \to \infty} a_n = 0$

            \item \textbf{ContraPositiva} Si $\lim_{n \to \infty} a_n \neq 0$ entonces la Serie es Divergente.
        \end{itemize}

    % =====================================================
    % ========       PRUEBA DE LA INTEGRAL        =========
    % =====================================================
    \clearpage
    \section{Prueba de la Integral}
        Suponga que f es una función:

        \begin{itemize}
            \item Continua
            \item Positiva
            \item Decreciente en $[1, \infty)$
        \end{itemize}

        y sea $a_n = f(n)$

        Entonces este criterio nos dira que:
        \begin{itemize}
           \item Si $\int_1^{\infty}f(x) dx$ es convergente, entonces $\sum_{n=1}^{\infty} a_n$ es convergente
           \item Si $\int_1^{\infty}f(x) dx$ es divergente, entonces $\sum_{n=1}^{\infty} a_n$ es divergente
        \end{itemize}

        Cuando use la prueba de la integral no es necesario iniciar la serie o la integral en $n=1$.
        Asimismo, no es necesario que $f(x)$ sea siempre decreciente.
        Lo importante es que $f(x)$ sea decreciente por último, es decir, decreciente para x más
        grande que algún número N. 



    % =====================================================
    % ========   CRITERIO DE COMPARACION DIRECTA  =========
    % =====================================================
    \clearpage
    \section{Criterio de Comparación: Directa}

        Supón que $a _n > 0$ y que también $b_n > 0$. Osea que ambos terminos siempre seran positivos.
        Entonces:

        \begin{itemize}
            \item Si $\Sigma b_n$ es convergente y $a_n \leq b_n$, entonces $a_n$ es convergente. 
            \item Si $\Sigma b_n$ es divergente y $a_n \geq b_n$, entonces $a_n$ es divergente. 
        \end{itemize}

        Naturalmente, al usar la prueba por comparación es necesario tener alguna serie conocida $\Sigma b_n$
        para los fines de la comparación. La mayor parte de las veces se usan las series:

        \begin{itemize}
            \item Series P: $\Sigma \frac{1}{n^p}$ que convergen si $p>1$ y divergen si $p\leq 1$
            \item Series P: $\Sigma ar^{n-1}$ que convergen si $|r|<1$ y divergen si $|r|\geq 1$
        \end{itemize}

        La condición $a_n \leq b_n$ o bien, $a_n \geq b_n$ de la prueba por comparación es para toda n, es
        necesario comprobar sólo que se cumple para $n \geq N$, donde N es un entero establecido, porque
        la convergencia de una serie no está afectada por un número finito de términos.

        % ==========================
        % ====    EJEMPLO  =========
        % ==========================
        \subsection{Ejemplo 1}
        Busquemos si la siguiente Serie diverge o converge:

        \begin{equation*}
            \sum_{n=1}^{\infty} \frac{5}{2n^2 +4n +3}
        \end{equation*}

        Ahora apliquemos el criterio de comparación: Podemos ver que esta serie se pacere mucho a esta
        que ya conocemos todos, a esta serie de ayuda la llamaremos $\Sigma b_n$:

        \begin{equation*}
            \sum b_n = \sum_{n=1}^{\infty} \frac{5}{2n^2} = \frac{5}{2}\sum_{n=1}^{\infty} \frac{1}{n^2}
        \end{equation*}

        Bueno, podemos decir que:
        \begin{equation*}
            \frac{5}{2n^2 +4n +3} < \frac{5}{2n^2}
        \end{equation*}
        Simplemente por el denominador.

        Y veamos que todo se cumplio, ademas sabemos que la serie $\Sigma \frac{1}{n^2}$ es convergente,
        entonces es seguro que la serie original que teniamos tambien lo sea. :D



    % =====================================================
    % ========   CRITERIO DE COMPARACION LIMITE   =========
    % =====================================================
    \clearpage
    \section{Criterio de Comparación: Limites}

        Supón que $a _n > 0$ y que también $b_n > 0$. Osea que ambos terminos siempre seran positivos.

        Entonces si:
        $\lim_{n \to \infty} \left( \frac{a_n}{b_n} \right) = L$

        (Donde obviamente L debe ser positivo y finito)

        Si todo esto se cumple entonces alguna de las dos proposiciones deben ser verdad:
        \begin{itemize}
            \item Ambas $\Sigma a_n$ y $\Sigma b_n$ divergen.
            \item Ambas $\Sigma a_n$ y $\Sigma b_n$ convergen.
        \end{itemize}

        % ==========================
        % ====    EJEMPLO  =========
        % ==========================
        \subsection{Ejemplo 1}
        Busquemos si la siguiente Serie diverge o converge:

        \begin{equation*}
            \sum_{n=1}^{\infty} \frac{3n^2+2}{(n^2-5)^2}
        \end{equation*}

        Antes que hacer nada, lo mejor es expandir:
        \begin{equation*}
            \sum_{n=1}^{\infty} \frac{3n^2+2}{n^4-10n^2+25}
        \end{equation*}
         
        Antes que seguir a nada, vemos si con la prueba de la divergencia podemos mostrar que diverge
        (para ahorrar trabajo):
        \begin{equation*}
            \lim_{n \to \infty} \frac{3n^2+2}{n^4-10n^2+25} = ?
        \end{equation*}

        Esto lo podemos calcular de muchas maneras, por ejemplo:
        \begin{equation*}
            \lim_{n \to \infty} \frac{ \frac{3n^2}{n^4} +\frac{2}{n^4}  }{ 1 - \frac{10n^2}{n^4} + \frac{25}{n^4} } = \frac{0}{1+0} = 0
        \end{equation*}

        Ok, paso esa prueba, lamentablemente esto no es
        suficiente para probar que converge.
        Ahora apliquemos el criterio de comparación: Podemos ver que esta serie se pacere mucho a esta que ya conocemos todos, a esta serie de ayuda la llamaremos $\Sigma b_n$:

        \begin{equation*}
            \sum b_n = \sum_{n=1}^{\infty} \frac{n^2}{n^4} = \sum_{n=1}^{\infty} \frac{1}{n^2}
        \end{equation*}

        Ahora aplicando lo que acabamos de ver:
        \begin{equation*}
            \lim_{n \to \infty} \left( \frac{a_n}{b_n} \right) = \left( \frac{ \frac{3n^2+2}{n^4-10n^2+25} }{ \frac{1}{n^2} } \right) =  \left( \frac{3n^4+2n^2}{n^4-10n^2+25} \right) = 3
        \end{equation*}

        Y veamos que todo se cumplio, 3 es finito y positivo y sabemos que la serie $\Sigma \frac{1}{n^2}$ es convergente, entonces es seguro que la serie original que teniamos tambien lo sea. :D


        % ==========================
        % ====    EJEMPLO  =========
        % ==========================
        \subsection{Ejemplo 2}
        Busquemos si la siguiente Serie diverge o converge:

        \begin{equation*}
            \sum_{n=1}^{\infty} \frac{n^{k-1}}{n^k+7}
        \end{equation*}
         
        Antes que seguir a nada, vemos si con la prueba de la divergencia podemos mostrar que diverge (para ahorrar trabajo):
        \begin{equation*}
            \lim_{n \to \infty} \frac{n^{k-1}}{n^k+7} = ?
        \end{equation*}

        Esto lo podemos calcular de muchas maneras, por ejemplo:
        \begin{equation*}
            \lim_{n \to \infty} \frac{n^{k-1}}{n^k+7} = \frac{ \frac{n^{k-1}}{n^k} }{1+\frac{7}{n^k}} = \frac{0}{1+0} = 0
        \end{equation*}

        Ok, paso esa prueba, lamentablemente esto no es suficiente para probar que converge, es más parece que debería diverger, así que probemos para eso:

        Ahora apliquemos el criterio de comparación, podemos ver que esta serie se pacere mucho a esta que ya conocemos todos, a esta serie de ayuda la llamaremos $\Sigma b_n$:

        \begin{equation*}
            \sum b_n = \sum_{n=1}^{\infty} \frac{n^{k-1}}{n^k} = \sum_{n=1}^{\infty} \frac{1}{n}
        \end{equation*}

        Sabemos que esta serie diverge.

        Ahora aplicando lo que acabamos de ver:
        \begin{equation*}
            \lim_{n \to \infty} \left( \frac{a_n}{b_n} \right) = \left( \frac{ \frac{n^{k-1}}{n^k+7} }{ \frac{1}{n} } \right) =  \left( \frac{n^k}{n^k+7} \right) = 1
        \end{equation*}

        Y veamos que todo se cumplio, 1 es finito y positivo y sabemos que la serie $\Sigma \frac{1}{n}$ es divergente, entonces es seguro que la serie original que teniamos tambien lo es :D


        % ==========================
        % ====    EJEMPLO  =========
        % ==========================
        \subsection{Ejemplo 3}
        Busquemos si la siguiente Serie diverge o converge:

        \begin{equation*}
            \sum_{n=1}^{\infty} \frac{3^n+2}{4^n-1}
        \end{equation*}

        Antes que seguir a nada, vemos si con la prueba de la divergencia podemos mostrar que diverge (para ahorrar trabajo). Esto se calcula muy facilmente porque el demoni- nador crece mucho mas rapidamente

        \begin{equation*}
            \lim_{n \to \infty}  \frac{3^n+2}{4^n-1} = 0
        \end{equation*}


        Ok, paso esa prueba, lamentablemente esto no es
        suficiente para probar que converge.
        Ahora apliquemos el criterio de comparación: Podemos ver que esta serie se pacere mucho a esta que ya conocemos todos, a esta serie de ayuda la llamaremos $\Sigma b_n$:

        \begin{equation*}
            \sum b_n = \sum_{n=1}^{\infty} \frac{3^n}{4^n} = \sum_{n=1}^{\infty} \left( \frac{3}{4} \right)^n 
        \end{equation*}

        Esto es una serie geometrica que converge, pues $|r| < 1$

        Ahora aplicando lo que acabamos de ver:
        \begin{equation*}
            \lim_{n \to \infty} \left( \frac{a_n}{b_n} \right) = \left( \frac{ \frac{3^n+2}{4^n-1} }{ \left( \frac{3}{4} \right)^n  } \right) =  \left( \frac{12^n + 2 \cdot 4^n}{12^n -3^n} \right) = \left( \frac{1 + 2 (\frac{1}{3})^n}{1 +  (\frac{1}{4})^n} \right) = 1
        \end{equation*}

        Y veamos que todo se cumplio, 1 es finito y positivo y sabemos que la serie $\Sigma (\frac{3}{4})^n$ es convergente, entonces es seguro que la serie original que teniamos tambien lo sea :D



    % =====================================================
    % ========       CRITERIO DE LA RAZON           =======
    % =====================================================
    \clearpage
    \section{Criterio de la Rázon}

        Sea una $\Sigma a_n$ una series de términos positivos, tal que:

        \begin{equation}
            \lim_{n \to \infty} \frac{a_n+1}{a_n} = L
        \end{equation}

        Entonces:
        \begin{itemize}
            \item $L < 1$ : La Serie Converge
            \item $L > 1$ : La Serie Diverge
            \item $L = 1$ : No nos dirá nada (cualquier serie P nos dará 1)
        \end{itemize}

        Pero si que podemos llegar a algo más:
        Si L da uno, podemos aplicar L' Hopital y volver a comprobar:

        \begin{equation}
        \frac{\frac{d}{dn} (a_{n+1})}{\frac{d}{dn} (a_n)} = L
        \end{equation}



        % ==========================
        % ====    EJEMPLO  =========
        % ==========================
        \subsection{Ejemplo 1}
        Busquemos si la siguiente Serie diverge o converge:

        \begin{equation*}
            \sum_{n=1}^{\infty} \frac{2^n}{n^2}
        \end{equation*}

        Probemos entonces la razón:
        \begin{equation*}
            \frac{ \frac{2^n}{n^2} }{ \frac{2 \cdot 2^n}{(n+1)^2 } } = \frac{ \frac{2^n}{n^2} }{ 2 \frac{2^n}{n^2 + 2n + 1} }
        \end{equation*}



    % =====================================================
    % ========       CRITERIO DE LAS ALTERNANTES    =======
    % =====================================================
    \clearpage
    \section{Criterio de las Series Alternantes}

        Para probar que una Serie Alternante $\sum_{n=1}^{\infty} (-1)^{n-1} b_n$ y $\sum_{n=1}^{\infty} (-1)^n b_n$ es convergente entonces tendrá que cumplir que:

        \begin{itemize}
            \item $\{b_n\}$ es una sucesión decreciente, es decir, $b_n \geq b_{n+1}$ para $n$ suficientemente grande
            \item Que el $\lim_{n \to \infty} b_n = 0$
        \end{itemize}

        Una observación es que este criterio solo sirve para demostrar convergencia, es decir, si alguna de las dos condiciones no se cumple sobre la serie alternante, no podemos concluir nada y será necesario usar otro criterio.

        % ==========================
        % ====    EJEMPLO  =========
        % ==========================
        \subsection{Ejemplo 1}
        Una sencilla para encaminarnos:
        \begin{equation*}
            \sum_{n=1}^{\infty} (-1)^{n-1} \frac{1}{n}
        \end{equation*}


         \begin{itemize}
            \item Paso 1: Limite $\lim_{n \to \infty} \frac{1}{n}=0$
            \item Paso 2: ¿Es Decreciente? Es decir :$\frac{1}{n}-\frac{1}{n+1} \geq 0 $
         \end{itemize}

        Como es verdadero entonces esta suma es convergente.





    % =====================================================
    % ========       CONVERGENCIA ABSOLUTA          =======
    % =====================================================
    \clearpage
    \section{Convergencia Absoluta}

        Sea $\{a_n\}$ una sucesión:

        \begin{itemize}
            \item Decimos que la serie $\sum_{n=1}^{\infty} a_n$ es \emph{Absolutamente Convergente} si la serie $\sum_{n=1}^{\infty} |a_n|$ converge.

            \item Si la serie $\sum_{n=1}^{\infty} a_n$ converge pero la serie $\sum_{n=1}^{\infty} |a_n|$ diverge, decimos que la serie es \emph{Condicionalmente Convergente}.
        \end{itemize}

        Podemos crear un Teorema muy interesante:
        Si $\sum_{n=1}^{\infty} a_n$ es absolutamente convergente, entonces también es convergente.

        El Teorema anterior es muy útil, ya que garantiza que una serie absolutamente convergente es convergente.
        Sin embargo, su recíproco no es necesariamente cierto: Las series que son Convergentes pueden o no ser Absolutamente Convergentes. 

        El ejemplo más famoso es la serie cuyo $n$-ésimo término es $a_n=\dfrac{(-1)^{n-1}}{n}$, ya que $\sum_{n=1}^{\infty}a_n$ converge por el teorema anterior, pero $\sum_{n=1}^{\infty} |a_n| = \sum_{n=1}^{\infty} \frac{1}{n}$ diverge por el criterio de las Series P.



% =====================================================
% ============        BIBLIOGRAPHY   ==================
% =====================================================
\clearpage
\bibliographystyle{plain}
	\begin{thebibliography}{9}

	% ============ REFERENCE #1 ========
	\bibitem{Sitio1} 
		ProbRob
		\\\texttt{Youtube.com}


	 

\end{thebibliography}



\end{document}