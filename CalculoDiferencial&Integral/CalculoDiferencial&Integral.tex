% ****************************************************************************************
% ********************    FUNCIONES, CALCULO DIFERENCIAL, INTEGRAL     *******************
% ****************************************************************************************


% =======================================================
% =======         HEADER FOR DOCUMENT        ============
% =======================================================
    % *********   DOCUMENT ITSELF   **************
    \documentclass[12pt]{report}                                    %Type of docuemtn and size of font
    \usepackage[margin=1.2in]{geometry}                             %Margins and Geometry pacakge
    \usepackage{ifthen}                                             %Allow simple programming
    \usepackage{hyperref}                                           %Create MetaData for a PDF and LINKS!
    \setlength{\parindent}{0pt}                                     %Eliminate ugly indentation
    \author{Oscar Andrés Rosas}                                     %Who I am

    % *********   LANGUAJE AND UFT-8   *********
    \usepackage[spanish]{babel}                                     %Please use spanish
    \usepackage[utf8]{inputenc}                                     %Please use spanish - UFT
    \usepackage[T1]{fontenc}                                        %Please use spanish
    \usepackage{textcmds}                                           %Allow us to use quoutes
    \usepackage{changepage}                                         %Allow us to use identate paragraphs

    % *********   MATH AND HIS STYLE  *********
    \usepackage{amsthm, amssymb, amsfonts, mathrsfs}                %Make math beautiful
    \usepackage[fleqn]{amsmath}                                     %Please make equations left
    \usepackage{centernot}                                          %Allow me to negate a symbol
    \decimalpoint                                                   %Use decimal point

    % *********   GRAPHICS AND IMAGES *********
    \usepackage{graphicx}                                           %Allow to create graphics
    \usepackage{wrapfig}                                            %Allow to create images
    \graphicspath{ {Graphics/} }                                    %Where are the images :D

    % *********   LISTS AND TABLES ***********
    \usepackage{listings}                                           %We will be using code here
    \usepackage[inline]{enumitem}                                   %We will need to enumarate
    \usepackage{tasks}                                              %Horizontal lists
    \usepackage{longtable}                                          %Lets make tables awesome
    \usepackage{booktabs}                                           %Lets make tables awesome
    \usepackage{tabularx}                                           %Lets make tables awesome
    \usepackage{multirow}                                           %Lets make tables awesome
    \usepackage{multicol}                                           %Create multicolumns

    % *********   HEADERS AND FOOTERS ********
    \usepackage{fancyhdr}                                           %Lets make awesome headers/footers
    \pagestyle{fancy}                                               %Lets make awesome headers/footers
    \setlength{\headheight}{16pt}                                   %Top line
    \setlength{\parskip}{0.5em}                                     %Top line
    \renewcommand{\footrulewidth}{0.5pt}                            %Bottom line

    \lhead{                                                         %Left Header
        \hyperlink{chapter.\arabic{chapter}}                        %Make a link to the current chapter
        {\normalsize{\textsc{\nouppercase{\leftmark}}}}             %And fot it put the name
    }

    \rhead{                                                         %Right Header
        \hyperlink{section.\arabic{chapter}.\arabic{section}}       %Make a link to the current chapter
            {\footnotesize{\textsc{\nouppercase{\rightmark}}}}      %And fot it put the name
    }

    \rfoot{\textsc{\small{\hyperref[sec:Index]{Ve al Índice}}}}    %This will always be a footer  

    \fancyfoot[L]{                                                  %Algoritm for a changing footer
        \ifthenelse{\isodd{\value{page}}}                           %IF ODD PAGE:
            {\href{https://compilandoconocimiento.com/yo/}          %DO THIS:
                {\footnotesize                                      %Send the page
                    {\textsc{Oscar Andrés Rosas}}}}                 %Send the page
            {\href{https://compilandoconocimiento.com}              %ELSE DO THIS: 
                {\footnotesize                                      %Send the author
                    {\textsc{Compilando Conocimiento}}}}            %Send the author
    }
    
    
    
% ========================================
% ===========   COMMANDS    ==============
% ========================================

    % =====  GENERAL TEXT  ==========
    \newcommand \Quote {\qq}                                        %Use: \Quote to use quotes
    \newenvironment{Indentation}[1][0.75em]                         %Use: \begin{Inde...}[Num]...\end{Inde...}
    {\begin{adjustwidth}{#1}{}}                                     %If you dont put nothing i will use 0.75 em
    {\end{adjustwidth}}                                             %This indentate a paragraph
    \newenvironment{SmallIndentation}[1][0.75em]                    %Use: The same that we upper one, just 
    {\begin{adjustwidth}{#1}{}\begin{footnotesize}}                 %footnotesize size of letter by default
    {\end{footnotesize}\end{adjustwidth}}                           %that's it
        
    % =====  GENERAL MATH  ==========
    \DeclareMathOperator \Space {\quad}                             %Use: \Space for a cool mega space
    \DeclareMathOperator \MiniSpace {\;}                            %Use: \Space for a cool mini space
    \newcommand \Such {\MiniSpace|\MiniSpace}                       %Use: \Such like in sets

    % =====  LOGIC  ==================
    \DeclareMathOperator \doublearrow {\leftrightarrow}             %Use: \doublearrow for a double arrow
    \newcommand \lequal {\MiniSpace \Leftrightarrow \MiniSpace}     %Use: \lequal for a double arrow
    \newcommand \linfire {\MiniSpace \Rightarrow \MiniSpace}        %Use: \lequal for a double arrow

    % =====  NUMBER THEORY  ==========
    \DeclareMathOperator \Naturals {\mathbb{N}}                     %Use: \Naturals por Notation
    \DeclareMathOperator \Integers {\mathbb{Z}}                     %Use: \Integers por Notation
    \DeclareMathOperator \Racionals{\mathbb{Q}}                     %Use: \Racionals por Notation
    \DeclareMathOperator \Reals {\mathbb{R}}                        %Use: \Reals por Notation
    \DeclareMathOperator \Complexs {\mathbb{C}}                     %Use: \Complex por Notation

    % === LINEAL ALGEBRA & VECTORS ===
    \DeclareMathOperator \LinealTransformation {\mathcal{T}}        %Use: \LinealTransformation for a cool T

    \newcommand{\pVector}[1]{                                       %Use: \pVector {Matrix Notation} use parentesis
        \ensuremath{\begin{pmatrix}#1\end{pmatrix}}                 %Example: \pVector{a\\b\\c} or \pVector{a&b&c} 
    }
    \newcommand{\lVector}[1]{                                       %Use: \lVector {Matrix Notation} use a abs 
        \ensuremath{\begin{vmatrix}#1\end{vmatrix}}                 %Example: \lVector{a\\b\\c} or \lVector{a&b&c} 
    }
    \newcommand{\Vector}[1]{                                        %Use: \Vector {Matrix Notation} no parentesis
        \ensuremath{\begin{matrix}#1\end{matrix}}                   %Example: \Vector{a\\b\\c} or \Vector{a&b&c}
    }


% =====================================================
% ============        COVER PAGE       ================
% =====================================================
\begin{document}
\begin{titlepage}

    \center
    % ============ UNIVERSITY NAME AND DATA =========
    \textbf{\textsc{\Large Proyecto Compilando Conocimiento}}\\[1.0cm] 
    \textsc{\Large Cálculo}\\[1.0cm] 

    % ============ NAME OF THE DOCUMENT  ============
    \rule{\linewidth}{0.5mm} \\[1.0cm]
        { \huge \bfseries Cálculo Diferencial e Integral}\\[1.0cm] 
    \rule{\linewidth}{0.5mm} \\[2.0cm]
    
    % ====== SEMI TITLE ==========
    {\LARGE Funciones, Límites, Derivadas e Integrales}\\[7cm] 
    
    % ============  MY INFORMATION  =================
    \begin{center} \large
    \textbf{\textsc{Autor:}}\\
    Rosas Hernandez Oscar Andres
    \end{center}

    \vfill

\end{titlepage}

% =====================================================
% ========                INDICE              =========
% =====================================================
\tableofcontents{}
\label{sec:Index}
\clearpage



% ////////////////////////////////////////////////////////////////////////////////////////////////////////////////////
% ////////////////////////////////////////   FUNCIONES Y LÍMITES     /////////////////////////////////////////////////
% ////////////////////////////////////////////////////////////////////////////////////////////////////////////////////
\part{Funciones y Límites}

    % ======================================================================================
    % ===========================         FUNCIONES               ==========================
    % ======================================================================================
    \chapter{Funciones}
        \clearpage

        % =====================================================
        % ==========    DEFINICION Y BASES           ==========
        % =====================================================
        \section{Definición}

            % ==================================
            % =========   FORMAL     ===========
            % ==================================
            \subsection*{Definición Formal}
                Usando lo que sabemos de relaciones (tengo un libro de eso :p) podemos recordar
                que las funciones no son mas que una relación entre dos conjuntos $f : A \to B$
                , donde se tiene que cumplir que para cada elemento de $A$ le corresponde un
                solo elemento de $B$.


            % ==================================
            % =========   ALTERNAS   ===========
            % ==================================
            \subsection*{Definiciones Alternas}
            Resulta útil pensar una función como si fuera una máquina.

            Si entra $x$ a la máquina, se acepta como una entrada y la máquina produce una
            salida $f(x)$ de acuerdo con la regla de la función. De este modo, puedes pensar
            el dominio como el conjunto de todas las entradas posibles y el rango como el
            conjunto de todas las salidas posibles.

            % ==================================
            % =====   IDEAS IMPORANTES   =======
            % ==================================
            \subsection{Ideas Importantes}

                Digamos que estamos hablando de una función $f$ cualquiera $f: A \to B$.


                \begin{itemize}
                    \item \textbf{Dominio}:
                        Solemos llamar dominio de $f$ al conjunto $A$.
                        \begin{equation}
                            Dominio = A
                        \end{equation}

                    \item \textbf{Rango}:
                        El rango de la función $f$ es simplemente todos los valores
                        de $f(x)$, o siendo mas exactos matemáticamente es $Rango \subseteq B$
                        donde tenemos que:
                        \begin{equation}
                            Rango = \{ f(x) \Such x \in A \} = f(A)
                        \end{equation}

                    \item \textbf{Variable Independiente}:
                        Llamamos variable independiente a un elemento cualquiera del conjunto $A$.
                        Generalmente usamos el símbolo $x$.

                    \item \textbf{Variable Dependiente}:
                        Llamamos variable independiente a un elemento cualquiera del conjunto $Rango$.
                        Generalmente usamos el símbolo $y$ ó $f(x)$.

                \end{itemize}


                

        % =====================================================
        % ========   CARACTERISTICAS DE FUNCIONES    ==========
        % =====================================================
        \section{Caracteristicas de Funciones}

            















% ////////////////////////////////////////////////////////////////////////////////////////////////////////////////////
% //////////////////////////////////       CALCULO DIFERENCIAL       /////////////////////////////////////////////////
% ////////////////////////////////////////////////////////////////////////////////////////////////////////////////////
\part{Cálculo Diferencial}
















% ////////////////////////////////////////////////////////////////////////////////////////////////////////////////////
% //////////////////////////////////       CALCULO INTEGRAL          /////////////////////////////////////////////////
% ////////////////////////////////////////////////////////////////////////////////////////////////////////////////////
\part{Cálculo Integral}

    % ======================================================================================
    % ========================  INTEGRACION IMPROPIA    ====================================
    % ======================================================================================
    \chapter{Integrales Impropias}
        \clearpage

        % =====================================================
        % ========         INTEGRACION IMPROPIA          ======
        % =====================================================
        \section{Integrales Impropias}

            Al definir la integral definida $\int_a^b f(x) dx$ estamos hablando
            de una función en la que:

            \begin{itemize}
                \item Esta definida en ese intervalo.
                \item No tiene una discontinuidad infinita
                \item Obviamente el intervalo es finito
            \end{itemize}

            Pero, que pasaría si no fuera así...

            Las integrales impropias explorar esta posibilidad así que veasmola:

        % ====================================================
        % ========== INTERVALOS INFINITOS ====================
        % ====================================================
        \clearpage
        \section{Tipo 1: Intervalos Infinitos}

            \subsubsection{Límite Superior}
            Si la $\int_a^t f(x) dx$ existe para todo número $t \geq a$, entonces
            lo siguiente es verdad, siempre que exista el límite (como un número finito).
            \begin{equation}
                \int_a^{\infty} f(x) dx = \lim_{t \to \infty} \int_a^t f(x) dx
            \end{equation}

            \subsubsection{Límite Inferior}
            Si la $\int_t^b f(x) dx$ existe para todo número $b \leq t$, entonces lo
            siguiente es verdad, siempre que exista el límite (como un número finito).
            \begin{equation}
                \int_{- \infty}^b f(x) dx = \lim_{t \to - \infty} \int_t^b f(x) dx
            \end{equation}

            \subsubsection{Convergencia}
            Las integrales impropias $\int_a^{\infty}f(x)dx$ y esta $\int_{-\infty}^bf(x)dx$
            se llaman \textbf{convegentes} si el límite existe y  \textbf{divergente} sino.

            \subsubsection{Ambos Límites}
            Si $\int_a^{\infty}f(x)dx$ y $\int_{-\infty}^bf(x)dx$ son convergentes, entonces
            se define esta asombrosa integral como:
            \begin{equation}
                \int_{-\infty}^{\infty} f(x) dx = \int_{-\infty}^{a} f(x) dx + \int_{a}^{\infty} f(x) dx   
            \end{equation}

            \subsubsection{Ejemplo}
            Podemos ver que con lo que sabemos ya podemos calcular la siguiente integral:

            \begin{equation*}
            \begin{split}
                \int_1^{\infty} \frac{1}{x^2} dx & = \lim_{t \to \infty} \int_1^t \frac{1}{x^2} dx \\
                & = \lim_{t \to \infty} \frac{-1}{x} \big\rvert_{1}^{t} \\
                & = \lim_{t \to \infty} \frac{-1}{t} - \frac{1}{-1} = \frac{-1}{t} + 1 = 1 + \frac{-1}{t} \\
                & = \lim_{t \to \infty} 1 + \frac{-1}{t} = 1 + 0 = 1
            \end{split}
            \end{equation*}

            \clearpage

        % ====================================================
        % ========== INTERVALOS INFINITOS ====================
        % ====================================================
        \clearpage
        \section{Tipo 2: Funciones Discontinuas}

            Si $f(x)$ es continua en $[a, b)$  pero discontinua en b, entonces
            (si el límite existe y es finito):
            \begin{equation}
                \int_a^b f(x) dx = \lim_{t \to b^-} \int_a^t f(x) dx
            \end{equation}

            Si $f(x)$ es continua en $(a, b]$  pero discontinua en a, entonces
            (si el límite existe y es finito):
            \begin{equation}
                \int_a^b f(x) dx = \lim_{t \to a^+} \int_t^b f(x) dx
            \end{equation}



            Si $\int_a^bf(x)dx$ es convergente, entonces se define esta asombrosa integral
            como (donde $c$ es $a<c<b$ ):
            \begin{equation}
                \int_a^b f(x) dx = \int_a^c f(x) dx + \int_c^b f(x) dx  
            \end{equation}




\end{document}